% Robert Adams CS 475

\documentclass[letterpaper,10pt]{article} %twocolumn titlepage 
\usepackage{graphicx}
\usepackage{amssymb}
\usepackage{amsmath}
\usepackage{amsthm}

\usepackage{alltt}
\usepackage{float}
\usepackage{color}
\usepackage{url}

\usepackage{balance}
\usepackage[TABBOTCAP, tight]{subfigure}
\usepackage{enumitem}
\usepackage{pstricks, pst-node}


\usepackage{geometry}
\geometry{margin=0.8in, textheight=8.5in} %textwidth=6in

%random comment

\newcommand{\cred}[1]{{\color{red}#1}}
\newcommand{\cblue}[1]{{\color{blue}#1}}

\usepackage{hyperref}

\def\name{Robert Adams}
%% The following metadata will show up in the PDF properties
\hypersetup{
	colorlinks = true,
	urlcolor = black,
	pdfauthor = {\name},
	pdfkeywords = {cs745},
	pdftitle = 	{CS 475 Project 8: OpenCL Array Multiplication},
pdfsubject = {CS 475 Project 8},
	pdfpagemode = UseNone,
}


\begin{document}
\title{CS 475 Project 8: OpenCL Array Multiplication} 
\author{Robert Adams}
\maketitle



\section{Commentary}

\indent First thing to note: GPU performance absolutely destroys SIMD performance,
iny my case by a factor of over 50x. 
\\
Also of note is that modifying local work size does not affect the 
performance. I was not able to see a difference on either my home 
system (specs below), or the CGEL lab. This may be because the video
cards of both are rather out of date.  For a new, or properly setup
machine we should see performance increase as the local size increases,
since more processing power has been assigned to the task.

\subsection{System Specs}

\begin{itemize}
\item AMD Athlon II X4 630 Processor, 2800 Mhz, 4 Cores, 4 Logical Processors
\item 4.00 GB RAM
\item  NVIDIA GeForce 98000 GT, 512MB RAM
\item  Driver Version 8.17.13
\end{itemize}


\pagebreak

\begin{figure} [ht]
	\centering
	\input{global.tex}
	%\caption{Speed of height calculations performed on a subdivided surface} 
	\label{runtimes}
\end{figure}

\begin{figure} [ht]
	\centering
	\input{local.tex}
	%\caption{Speed of height calculations performed on a subdivided surface} 
	\label{runtimes}
\end{figure}

%\begin{table}  [ht]
	%\centering
	%    \begin{tabular}{llllll}
%\verb|#|months & precipitation, in. & temperature, celsuis & height, in.
               %& \#deer & blood rain\\ \hline 
%0 & 7.130446 & 3.047354 & 7.280658 & 1 & 0\\ 
%1 & 11.88923 & 5.459258 & 7.399446 & 2 & 0\\ 
%2 & 12.868473 & 5.293376 & 7.097805 & 3 & 0\\ 
%3 & 11.705143 & 13.475433 & 0 & 4 & 0\\ 

		%    \end{tabular}
	%\end{table}

	\end{document}
